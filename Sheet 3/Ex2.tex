\section*{Excercise 2}
\subsection*{a)}
\begin{figure}[H]
  \centering
  \begin{tikzpicture}[scale=1.33]
  \draw[->] (-1.5,0)--(1.5,0);
  \draw[->] (0,-1.5)--(0,1.5);
  \node at (0.25,-0.25) {0};
  \draw (1,0)--(0,1) -- (-1,0) -- (0,-1) -- (1,0);
  \node at (1,-0.25){1};
  \node at (-1,-0.25){-1};
  \node at (0.25,1){1};
  \node at (0.25,-1){-1};
  \end{tikzpicture}
  \caption{$B^2_1$}
\end{figure}

In $\mathbb{R}^3$ $B^3_1$ is a double pyramid.
It's corners are located weher exactly one entry of the point is 1 or -1 and all others 0.
\subsection*{b)}
\[\text{vol}(B^2_1) = 2\frac{1}{2}\cdot2\cdot1 = 2\]
\[\text{vol}(B^3_1) = 2\frac{1}{3}\cdot\sqrt[2]{2}^2\cdot1 = \frac{4}{3}\]
\subsection*{c)}
We can derive theat $vol(B^n_1) = 2\frac{1}{n}\sqrt[n-1]{2}^{n-1}$.
\[\lim_{n \to \infty} vol(B^n_1) = 2 \lim_{n \to \infty} \frac{1}{n}\cdot2 = 0\]