\section*{Exercise 3}
We know that for a hypercube of dimension $d$ and edge length $l$ the spacial diagonal has length \[s = l\sqrt{d}\].
We also have the radius of the outer hyperballs given as \[r_o=\frac{l}{4}\].
So the radius of the inner hyperball is half the difference of the spacial diagonal and two times the diameter of the outer hyperballs.
That leads to \[r_i = \frac{1}{2}s - 4 * r_o = \frac{1}{2}l\sqrt{d}-l\].
Using this measure we can we can require the diameter of the inner hyperball to be larger than the sides of the hypercube.
\[l\sqrt{d}-l > l\]
which resolves to \[d>4\].

So a dimension of at least five results in the hyperball 'sticking out'.